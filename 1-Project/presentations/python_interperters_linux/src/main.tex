\documentclass[10pt,compress]{beamer}
    \useoutertheme{miniframes}
    \usepackage[english]{babel}

    \usepackage{amsmath}
    \usepackage{amsfonts} 
    \usepackage{amssymb}
    \usepackage{amsthm}
    \usepackage{mathtools}

    \usepackage[utf8]{inputenc}

    \usepackage{float}
    \usepackage{graphicx}
    \usepackage{caption}
    \usepackage{subcaption}

    \graphicspath{{./src/figures/}}

    %\usepackage{fancyhdr} %custom headers and footers layout
    \usepackage{lastpage} %package to print the last page
    %\pagestyle{fancy} %fancy page style

    \usepackage{textcomp} 
    \usepackage{multicol} 
    \usepackage{multirow}

    \usepackage[table]{xcolor}
    \usepackage{booktabs}

    \usepackage[backend=biber,
    bibstyle=ieee, 
    citestyle=numeric-comp,
    natbib=true,
    doi=false, 
    url=false,
    isbn=false,
    mincitenames=1,
    maxcitenames=1,
    minbibnames=1,
    maxbibnames=99,
    backref=false,]
    {biblatex}
    \addbibresource[label=main]{./src/references.bib}

    \usepackage{url}
    \usepackage{hyperref}

    %\usetheme{metropolis}

    %edit the properties of your PDF documents which will be displayed
    \hypersetup{
        bookmarks=true, 		% show bookmarks bar?
        unicode=true,  		% non-Latin characters in Acrobat’s bookmarks
        pdftoolbar=true,        % show Acrobat’s toolbar?
        pdfmenubar=true,        % show Acrobat’s menu?
        pdffitwindow=true,      % page fit to window when opened
        pdftitle={},    % title
        pdfauthor={M. Skocic},     % author
        pdfsubject={},   % subject of the document
        pdfnewwindow=true,      % links in new window
        pdfkeywords={}, % list of keywords
        colorlinks=false,       % false: boxed links; true: colored links
        linkcolor=red,          % color of internal links
        citecolor=green,        % color of links to bibliography
        filecolor=magenta,      % color of file links
        urlcolor=cyan           % color of external links
    }

    \usepackage{tikz}
    \usepackage{circuitikz}
    \usetikzlibrary{decorations.pathmorphing,arrows,calc}

\title{Multiple Python interpreters on Linux}
\author{M. Skocic}
\date{}

\begin{document}

\begin{frame}
    \titlepage
\end{frame}

\section{Dependencies}
\begin{frame}{Required packages}
    
    \begin{alertblock}{Debian-based distributions}
        \begin{semiverbatim}
        {\small
        \$ sudo apt-get install build-essential checkinstall
        libreadline-dev libncursesw5-dev libssl-dev 
        libsqlite3-dev tk-dev libgdbm-dev libc6-dev libbz2-dev libffi-dev
        zlib1g-dev liblzma-dev libgdbm-compact-dev libnsl-dev}
        \end{semiverbatim}
    \end{alertblock}

    \begin{alertblock}{Other distributions}
        Please refer to your distribution documentation for finding the 
        equivalent packages.
    \end{alertblock}

\end{frame}


\section{Build}
\begin{frame}{Compilation}
    \begin{alertblock}{Example with Python 3.13}
        \begin{semiverbatim}
            {\small

            \$ wget https://www.python.org/ftp/python/3.13.0/python-3.13.0.tgz
            
            \$ tar -xvf python-3.13.0.tgz
            
            \$ cd python-3.13.0

            \$ sudo ./configure --enable-optimizations

            \$ make

            \$ sudo make altinstall
            }
        \end{semiverbatim}
    \end{alertblock}
    
    \begin{alertblock}{Free threaded with Python 3.14}
        Python 3.14 allows to disable the GIL.
        \begin{semiverbatim}
            {\small

            \$ sudo ./configure --enable-optimizations --disable-gil

            }
        \end{semiverbatim}
    \end{alertblock}
\end{frame}

\section{Installation}
\begin{frame}{Alternatives}
    \begin{alertblock}{Update}
        \begin{semiverbatim}
            {\small

            \$ sudo update-alternatives --install \ 

            /usr/bin/python3 python /usr/bin/python3.13 1 \
            
            --slave /usr/bin/pip3 pip /usr/bin/pip3.13

            \$ update-alternatives --config python
            }
        \end{semiverbatim}
    \end{alertblock}
    
    \begin{alertblock}{Configure}
        \begin{semiverbatim}
            {\small
            \$ update-alternatives --config python
            }
        \end{semiverbatim}
    \end{alertblock}
\end{frame}

\end{document}
