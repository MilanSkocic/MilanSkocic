\section{Estimation of the confidence intervals}

The scalar function to be minimized defined in \ref{eq:D} used for fitting 
the photocurrent spectra ensures a fairly fast convergence towards the $3m$ 
parameters defining the semiconductive contributions. However, the as-defined 
scalar function could not be used for estimating the confidence intervals. 
An alternative scalar function was defined to be computed with the optimal 
parameters in order to estimate the confidence intervals. 
The statistics of curve fitting shows that the confidence intervals can be 
estimated using the least-squares method which can be applied to nonlinear 
systems \citep{bevington2003,nocedal2006}.

The least-squares regression uses the proprieties of the $\chi ^2$ distribution. 
Consequently, of the experimental measurements of the photocurrent spectra are 
assumed to follow the normal distribution. Moreover, the least-squares method 
can be strictly applied only when the experimental variances are known for 
each energy value of the photocurrent spectrum. Nonetheless, the latter are 
not always known as it is the case for the photoelectrochemical characterizations. 
Consequently, modifications of the relationships defined for the ideal 
situation were necessary.

\subsection{Ideal Situation}
The equation \ref{eq:chi2} presents the scalar function $\chi ^2$ defined in 
the least-squares method when the experimental variances $\sigmae ^2$ are known. 
The residuals, $\epsilon$, weighted by the inverse of variances are given by 
the equation \ref{eq:epsilon}. $\chi ^2$ is therefore defined as the sum of 
the weighted residuals as illustrated by the equation \ref{eq:chi2_epsilon}

\begin{equation}
\begin{split}
\chi ^2 = \sum _{\hv} \frac{\modi{\iphc-\iphe}^2}{\sigmae^2}
\end{split}
\label{eq:chi2}
\end{equation}

\begin{equation}
\begin{split}
\epsilon = \frac{\modi{\iphc-\iphe}^2}{\sigmae}
\end{split}
\label{eq:epsilon}
\end{equation}

\begin{equation}
\begin{split}
\chi ^2 = \sum _{\hv} \eps ^2
\end{split}
\label{eq:chi2_epsilon}
\end{equation}

$\gradX$ approaches zero when the parameter values are approaching the optimum values. 
The covariance matrix of the fitted parameters, $\sigmap ^2$, can be estimated 
with the Jacobian, $\je$, of the weighted residuals \citep{bevington2003} and 
its expression is given by the equation \ref{eq:sigma_p}. For nonlinear systems, 
as it is the case for the photocurrent, the equation \ref{eq:sigma_p} is a 
first-order approximation. In fact, the approximation is valid because the 
second-order terms are close to zero which avoids to compute the Hessian \citep{press2007}.

\begin{equation}
\begin{split}
\sigmap = ( \je ^T \cdot \je ) ^{-1}
\end{split}
\label{eq:sigma_p}
\end{equation}

The diagonal terms of the covariance matrix represent the variances of the parameters. 
The P\% confidence interval of the parameters, $CI_{P\%}$ is obtained by 
multiplying the standard deviations with the student coefficient, $\tvp$, 
with dof being the degree of freedom corresponding to number of experimental 
points of the photocurrent, N, minus the number of parameter, 3m. 
The probability $P\%$ was set to 95\%. The confidence intervals of the parameters 
are given by the equation \ref{eq:CIP}.

\begin{equation}
\begin{split}
CI_P &= \sqrt{ diag(\sigmap ^2)} \cdot \tvp \\
dof &= N-3m
\end{split}
\label{eq:CIP}
\end{equation}

\subsection{Real Situation}
The confidence interval can be estimated even when the experimental variances, 
$\sigmae ^2$, are not known. However, it is necessary to modify the equations 
presented in section 3.1. The objective is to define a scalar function, $S$, 
which behaves like $\chi ^2$ with a constant scaling factor $g$. 
The scalar function, $S$, is defined using real and positive weighting terms, 
$w$, as shown in equation \ref{eq:chi2_epsilon}. Similarly to the Equation 6, 
the $S$ function corresponds to the sum of the weighted residuals $\epsp$.

\begin{equation}
\begin{split}
S &= \sum _{\hv} \frac{\modi{\iphc-\iphe}^2}{w ^2} \\
%%\epsp &= \sum _{\hv} \frac{\modi{\iphc-\iphe}}{\sqrt{w}} \\
%S & = \sum _{\hv} \epsp ^2 
\end{split}
\label{eq:S}
\end{equation}

The weighting terms are defined in order to isolate the scaling factor, $g$, 
and the experimental variances such as illustrated by the equation \ref{eq:w}. 
Therefore, the weighting terms are considered proportional to the experimental 
variances.
\begin{equation}
	w=g \cdot \frac{1}{\sigmae ^2}
	\label{eq:w}
\end{equation}	
	
Combining equation \ref{eq:S} and equation \ref{eq:w}, the scalar function $S$ 
becomes proportional to $\chi ^2$. Moreover, $\frac{\chi ^2}{\nu}$ goes to unity 
when the optimum values of the parameters are reached. Consequently, 
the scaling factor $g$ can be computed with the optimum value of $S$ as shown 
by the equation \ref{eq:S_scaling_unity}.
\begin{equation}
\begin{split}
S &=g \cdot \chi ^2 \\
\frac{S}{\nu}&=g \cdot \frac{\chi ^2}{\nu} \simeq 1
\end{split}
\label{eq:S_scaling_unity}
\end{equation}
	

The covariance matrix, $\sigmap$, can therefore be estimated with the scaling 
factor $g$ and the Jacobian of the weighted residuals, $\jep$ as illustrated 
by the equation \ref{eq:sigma_p}. The confidence intervals are then computed 
using the Equation \ref{eq:CIP}.
	
\begin{equation}
\begin{split}
\sigmap ^{\prime 2} &= ( \jep ^T \cdot \jep ) ^{-1} \\
\sigmap &= g \cdot \sigmap ^2
\end{split}
\label{eq:sigma_p}
\end{equation}

The choice of the weighted terms was made considering that the experimental 
variances are proportional to average noise of the photocurrent modulus in 
dark conditions, $\bar{\epsilon _d}$. It has been proposed that the variances 
are smaller when the quantum yield is greater. 
The latter is represented by the photocurrent corrected by the photon flux 
normalized to its maximum value, $I_{ph,N}$. 
The normalization ensures that the weighting terms have the same dimension as 
the inverse of the variances as shown in equation \ref{eq:w_epsd}. 
$S$ is therefore adimensional as it is the case for $\chi ^3$. 
The Jacobian, $\jep$ is numerically estimated by fixing the finite difference 
step to the squared root of the machine precision \citep{nocedal2006, press2007}.

\begin{equation}
\begin{split}
\sigmae ^2 & \propto \frac{\bar{\epsilon _d}}{\vert I_{ph,N} \vert} \\
w = & \frac{1}{\sigmae ^2} = g \cdot \frac{\vert I_{ph,N} \vert ^2}{\bar{\epsilon _d} ^2}
\end{split}
\label{eq:w_epsd}
\end{equation}
