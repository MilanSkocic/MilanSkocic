\section{Basics}\label{sec_basics}

PEC takes advantage of the photovoltaic effect, discovered by 
\citet{becquerel1839-1} in 1839, that occurs at the interface of a semiconductor
and an electrolyte. 
In fact, the first experience showed the occurrence of a photopotential and 
a photocurrent under illumination when a silver electrode, 
covered with an oxide layer, was immersed in an acidic medium and connected 
to a platinum electrode. 
Nonetheless, the first studies focused on the understanding of the interfacial 
processes were performed much later 
\citep{stimming1986-1,gerischer1966-1,copeland1942-1}.


The basics of photoelectrochemistry and application examples are presented in 
the following sections and they are largely described in the literature 
\citep{morrison1980-1,gerischer1985-1,memming2008-1,marcus2006-1,bard2002-1,sato1998-1}. 
Several hypotheses are needed in order to apply the theoretical concepts:  
\begin{itemize}
\item semiconductor are considered to be ideal i.e. crystallized and homogeneous  
\item the dielectric constant of the semiconductor is independent of the light wavelength  
\item the capacity of the Helmholtz layer is greater than the capacitance of the space charge capacitance  
\item the potential drop in the Helmholtz layer is independent of the applied potential and is negligible
\end{itemize}

The hypotheses are rarely fully respected in the case of oxides or passive 
films formed on industrial alloys. Nonetheless, the literature shows that the 
developed models can be applied to non-ideal systems such as oxides 
and passive films.


\subsection{Electronic Band Structure}

Solids are generally classified into three groups: 
conductors, semiconductors and isolators. 
Each category can be illustrated with a specific band structure as shown in 
figure~\ref{fig_band_model}. 
Valence and conduction bands correspond to allowed energy states for the electrons. 
The lowest energy level of the conduction band is labeled $E_c$ and the 
highest energy level of the valence band is labeled $E_v$. 
They are separated by a band gap, $E_g$, with no allowed energy states. 
The repartition of the electrons among both bands are described by the position 
of the Fermi Level, $E_F$, which represents the highest energy state that 
can be occupied level at 0K. 
It is equivalent to the electrochemical potential in solid phases.

\begin{figure}[h]
\centering
\begin{circuitikz}[scale=1.0]
\small
\draw[->] (0,-3) -- ++(0,6);
\node[anchor=center, rotate=90] at (-0.25,0) {Electron Energy};

\draw[dashed, thick] (0,0) -- ++(10,0);
\node[draw=none, anchor=south] at (5,0.2) {Fermi Level: $E_F$};

\node[rectangle, minimum height=1.5cm, anchor=south, draw=red, align=center, text width=2cm, fill=red!20] at (2, -0.2) {Conduction Band};
\node[rectangle, minimum height=1.5cm, anchor=north, draw=green, align=center, text width=2cm, fill=green!20] at (2, 0.2) {Valence Band};
\node[rectangle, minimum height=0.4cm, anchor=center, draw=orange, align=center, text width=2cm, fill=orange!20] at (2, 0.0) {};

\node[rectangle, anchor=south, draw=red, align=center, text width=2cm, fill=red!20] at (5, 1.) {Conduction Band};
\node[rectangle, anchor=north, draw=green, align=center, text width=2cm, fill=green!20] at (5, -1.) {Valence Band};

\node[draw=none, anchor=south west, blue] at (9.2,+0.2) {Band Gap: $E_g$};
\draw[<->, thick, blue,] (9,-1.5) -- ++(0,3);
\node[rectangle, anchor=south , draw=red, align=center, text width=2cm, fill=red!20] at (9, 1.5) {Conduction Band};
\node[rectangle, anchor=north , draw=green, align=center, text width=2cm, fill=green!20] at (9, -1.5) {Valence Band};

\end{circuitikz}
\caption{Schematic representation of the electronic band structure \citep{marucco2006-1}: 
a) conductor, b) semiconductor, c) isolator}
\label{fig_band_model}
\end{figure}

The electronic conduction is due to the movement either of the negatively 
charged electrons in the conduction band or of the positively charged holes 
in the valence band or both simultaneously. 
Consequently, the conduction depends on the number of available charge carriers
in the conduction band and in the valence band. 
In conductors, an overlap of the conduction and the valence bands occurs 
which means that the highest allowed energy band is partially filled. 
The distinction between a semiconductor and an isolator is less obvious 
because the conduction depends on the band gap and the energy provided by 
the environment to the electrons from the valence band in order to jump 
into the conduction band.

In semiconductors, charge carriers can be generated by three mechanisms: 
\emph{thermal excitation, photoexcitation and doping} as shown in 
figure~\ref{fig_excitation_carrier}. 
In the case of very low band gaps, thermal excitation can be enough in order 
to eject an electron from the valence band to the conduction band. 
Photoexcitation ejects electrons from the valence band to the conduction 
band when an incident photon, with energy greater than the band gap, is absorbed. 
Doping introduces additional energy level located in between the conduction and 
valence bands.

Doping occurs when the stoichiometry is altered or when impurities are 
introduced in the crystallographic lattice of the semiconductor. 
In the case of n-type semiconductors, the donor energy levels $E_d$ lie just 
under the conduction band. The electrons from the donor levels are ejected by 
thermal excitation. 
Consequently, the majority charge carriers are negatively charged electrons 
in the band conduction. 
Similarly, the acceptor energy levels $E_a$, of p-type semiconductors, 
lie just above the band valence. 
The latter trap electrons from the valence band and therefore create holes. 
Consequently, the majority charge carriers are positively charged holes.

\begin{figure}[h]
\centering
\begin{circuitikz}[scale=1.0]

% Thermal Excitation
\coordinate (XY) at (0,0);
\draw[-Stealth, thick] ($(XY)+(1,0)$) -- ++(0,3);
\draw[color=green] ($(XY)+(0,0)$) -- ++(2,0);
\draw[rectangle, anchor=north , draw=green, fill=green!20] ($(XY)+(0.2,0)$) rectangle ++(1.6,-1);
\draw ($(XY)+(1,-0.5)$) circle [radius=0.3] node {+};
\draw[color=red] ($(XY)+(0,3)$) -- ++(2,0);
\draw[rectangle, anchor=north , draw=red, fill=red!20] ($(XY)+(0.2,3)$) rectangle ++(1.6,1);
\draw ($(XY)+(1,3.5)$) circle [radius=0.3] node {-};
\node[anchor=west] at ($(XY)+(2,0)$) {$E_v$};
\node[anchor=west] at ($(XY)+(2,3)$) {$E_c$};
\node[anchor=center, align=center] at ($(XY)+(1, -2)$) {a) Thermal\\Excitation};

% Photoexcitation
\coordinate (XY) at (3,0);
\draw[-Stealth, thick] ($(XY)+(1,0)$) -- ++(0,3);
\draw[color=green] ($(XY)+(0,0)$) -- ++(2,0);
\draw[rectangle, anchor=north , draw=green, fill=green!20] ($(XY)+(0.2,0)$) rectangle ++(1.6,-1);
\draw ($(XY)+(1,-0.5)$) circle [radius=0.3] node {+};
\draw[color=red] ($(XY)+(0,3)$) -- ++(2,0);
\draw[rectangle, anchor=north , draw=red, fill=red!20] ($(XY)+(0.2,3)$) rectangle ++(1.6,1);
\draw ($(XY)+(1,3.5)$) circle [radius=0.3] node {-};
\node[anchor=west] at ($(XY)+(2,0)$) {$E_v$};
\node[anchor=west] at ($(XY)+(2,3)$) {$E_c$};
\node[anchor=center, align=center] at ($(XY)+(1, -2)$) {b) Photoexcitation};

% Doping
\coordinate (XY) at (6,0);
\draw[color=green] ($(XY)+(0,0)$) -- ++(1.5,0);
\draw[rectangle, anchor=north , draw=green, fill=green!20] ($(XY)+(0.2,0)$) rectangle ++(1.1,-1);
\draw[color=red] ($(XY)+(0,3)$) -- ++(1.5,0);
\draw[rectangle, anchor=north , draw=red, fill=red!20] ($(XY)+(0.2,3)$) rectangle ++(1.1,1);
\node[anchor=north, align=center] at ($(XY)+(0.75, -1)$) {n-type};
\draw[color=black] ($(XY)+(0,2)$) -- ++(1.5,0);
\draw[-Stealth, thick] ($(XY)+(0.5,2.5)$) -- ++(0,0.5);
\draw[-Stealth, thick] ($(XY)+(1,2.5)$) -- ++(0,0.5);
\node[draw, anchor=south, align=center, circle, minimum size=0.2pt, inner sep=0] at ($(XY)+(0.5,2)$) {+};
\node[draw, anchor=south, align=center, circle, minimum size=0.2pt, inner sep=0] at ($(XY)+(1,2)$) {+};
\node[draw, anchor=south, align=center, circle, minimum size=0.2pt, inner sep=2] at ($(XY)+(0.5,3)$) {-};
\node[draw, anchor=south, align=center, circle, minimum size=0.2pt, inner sep=2] at ($(XY)+(1,3)$) {-};
\node[anchor=west] at ($(XY)+(1.5,0)$) {$E_v$};
\node[anchor=west] at ($(XY)+(1.5,3)$) {$E_c$};
\node[anchor=west] at ($(XY)+(1.5,2.0)$) {$E_d$};

\node[anchor=center, align=center] at ($(XY)+(2, -2)$) {c) Doping};

\coordinate (XY) at (8.2,0);
\draw[color=green] ($(XY)+(0,0)$) -- ++(1.5,0);
\draw[rectangle, anchor=north , draw=green, fill=green!20] ($(XY)+(0.2,0)$) rectangle ++(1.1,-1);
\draw[color=red] ($(XY)+(0,3)$) -- ++(1.5,0);
\draw[rectangle, anchor=north , draw=red, fill=red!20] ($(XY)+(0.2,3)$) rectangle ++(1.1,1);
\node[anchor=north, align=center] at ($(XY)+(0.75, -1)$) {p-type};
\draw[color=black] ($(XY)+(0,1)$) -- ++(1.5,0);
\draw[-Stealth, thick] ($(XY)+(0.5,0.0)$) -- ++(0,1);
\draw[-Stealth, thick] ($(XY)+(1,0.0)$) -- ++(0,1);
\node[draw, anchor=south, align=center, circle, minimum size=0.2pt, inner sep=0] at ($(XY)+(0.5,-0.5)$) {+};
\node[draw, anchor=south, align=center, circle, minimum size=0.2pt, inner sep=0] at ($(XY)+(1,-0.5)$) {+};
\node[draw, anchor=south, align=center, circle, minimum size=0.2pt, inner sep=2] at ($(XY)+(0.5,1)$) {-};
\node[draw, anchor=south, align=center, circle, minimum size=0.2pt, inner sep=2] at ($(XY)+(1,1)$) {-};
\node[anchor=west] at ($(XY)+(1.5,0)$) {$E_v$};
\node[anchor=west] at ($(XY)+(1.5,3)$) {$E_c$};
\node[anchor=west] at ($(XY)+(1.5,1)$) {$E_a$};
\end{circuitikz}
\caption{Schematic representation of the mechanisms generating charge carriers in semiconductors \citep{finklea1983-1}: 
a) thermal excitation, b) photoexcitation, c) doping}
\label{fig_excitation_carrier}
\end{figure}

The Fermi level $E_F$ in intrinsic semiconductors is located at the mid-gap. 
The n-type and p-type doping shift the Fermi level towards band edges 
$E_c$ and $E_v$, respectively. 
The figure~\ref{fig_fermi_position} shows the position of the Fermi level 
with respect to the semiconductor type.

\begin{figure}[H]
\centering
\begin{circuitikz}[scale=1.0]
% Intrinsic
\coordinate (XY) at (0,0);
\draw[color=green] ($(XY)+(0,0)$) -- ++(2,0);
\draw[rectangle, anchor=north , draw=green, fill=green!20] ($(XY)+(0.2,0)$) rectangle ++(1.6,-1);
\draw[color=red] ($(XY)+(0,3)$) -- ++(2,0);
\draw[rectangle, anchor=north , draw=red, fill=red!20] ($(XY)+(0.2,3)$) rectangle ++(1.6,1);
\draw[color=black, dashed, thick] ($(XY)+(0,1.5)$) -- ++(2,0);
\node[anchor=west] at ($(XY)+(2,0)$) {$E_v$};
\node[anchor=west] at ($(XY)+(2,3)$) {$E_c$};
\node[anchor=west] at ($(XY)+(2,1.5)$) {$E_F$};
\node[anchor=center, align=center] at ($(XY)+(1, -2)$) {a) Intrinsic};

% n-type
\coordinate (XY) at (4,0);
\draw[color=green] ($(XY)+(0,0)$) -- ++(2,0);
\draw[rectangle, anchor=north , draw=green, fill=green!20] ($(XY)+(0.2,0)$) rectangle ++(1.6,-1);
\draw[color=red] ($(XY)+(0,3)$) -- ++(2,0);
\draw[rectangle, anchor=north , draw=red, fill=red!20] ($(XY)+(0.2,3)$) rectangle ++(1.6,1);
\draw[color=black, dashed, thick] ($(XY)+(0,2.5)$) -- ++(2,0);
\draw[color=black,  thick] ($(XY)+(0,2.0)$) -- ++(2,0);
\node[anchor=west] at ($(XY)+(2,0)$) {$E_v$};
\node[anchor=west] at ($(XY)+(2,3)$) {$E_c$};
\node[anchor=west] at ($(XY)+(2,2.5)$) {$E_F$};
\node[anchor=west] at ($(XY)+(2,2.0)$) {$E_d$};
\node[anchor=center, align=center] at ($(XY)+(1, -2)$) {b) n-type};

% p-type
\coordinate (XY) at (8,0);
\draw[color=green] ($(XY)+(0,0)$) -- ++(2,0);
\draw[rectangle, anchor=north , draw=green, fill=green!20] ($(XY)+(0.2,0)$) rectangle ++(1.6,-1);
\draw[color=red] ($(XY)+(0,3)$) -- ++(2,0);
\draw[rectangle, anchor=north , draw=red, fill=red!20] ($(XY)+(0.2,3)$) rectangle ++(1.6,1);
\draw[color=black, dashed, thick] ($(XY)+(0,0.5)$) -- ++(2,0);
\draw[color=black,  thick] ($(XY)+(0,1.0)$) -- ++(2,0);
\node[anchor=west] at ($(XY)+(2,0)$) {$E_v$};
\node[anchor=west] at ($(XY)+(2,3)$) {$E_c$};
\node[anchor=west] at ($(XY)+(2,0.5)$) {$E_F$};
\node[anchor=west] at ($(XY)+(2,1.0)$) {$E_a$};
\node[anchor=center, align=center] at ($(XY)+(1, -2)$) {b) p-type};

\end{circuitikz}
\caption{Schematic representation of the Fermi level with respect to the 
semiconduction type \citep{finklea1983}: a) intrinsic, b) n-type, c) p-type.}
\label{fig_fermi_position}
\end{figure}




\subsection{Semiconductor/electrolyte interface in dark condition}




\subsection{Semiconductor/electrolyte interface under illumination}