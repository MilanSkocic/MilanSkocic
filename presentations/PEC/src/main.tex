\documentclass[10pt,compress]{beamer}
\useoutertheme{miniframes}
\usepackage[english]{babel}
%\usepackage[margin=1in]{geometry}

\usepackage{amsmath}
\usepackage{amsfonts} 
\usepackage{amssymb}
\usepackage{amsthm}
\usepackage{mathtools}

\usepackage[utf8]{inputenc}
%\usepackage[exscale, amsfonts, amssymb]{concmath}
%\renewcommand*{\bfseries}{\mdseries}

\usepackage{float}
\usepackage{graphicx}
\usepackage{caption}
\usepackage{subcaption}

\graphicspath{{./src/figures/}}

%\usepackage{fancyhdr} %custom headers and footers layout
\usepackage{lastpage} %package to print the last page
%\pagestyle{fancy} %fancy page style

\usepackage{textcomp} 
\usepackage{multicol} 
\usepackage{multirow}

\usepackage[table]{xcolor}
\usepackage{booktabs}

\usepackage[backend=biber,
bibstyle=ieee, 
citestyle=numeric-comp,
natbib=true,
doi=false, 
url=false,
isbn=false,
mincitenames=1,
maxcitenames=1,
minbibnames=1,
maxbibnames=99,
backref=false,]
{biblatex}
\addbibresource[label=main]{./src/references.bib}

\usepackage{url}
\usepackage{hyperref}

%edit the properties of your PDF documents which will be displayed
\hypersetup{
    bookmarks=true, 		% show bookmarks bar?
    unicode=true,  		% non-Latin characters in Acrobat’s bookmarks
    pdftoolbar=true,        % show Acrobat’s toolbar?
    pdfmenubar=true,        % show Acrobat’s menu?
    pdffitwindow=true,      % page fit to window when opened
    pdftitle={PhotoElectrochemistry --- Theoretical Background},    % title
    pdfauthor={M. Skocic},     % author
    pdfsubject={},   % subject of the document
    pdfnewwindow=true,      % links in new window
    pdfkeywords={}, % list of keywords
    colorlinks=false,       % false: boxed links; true: colored links
    linkcolor=red,          % color of internal links
    citecolor=green,        % color of links to bibliography
    filecolor=magenta,      % color of file links
    urlcolor=cyan           % color of external links
}

\usepackage{tikz}
\usepackage{circuitikz}
\usetikzlibrary{decorations.pathmorphing,arrows,calc}

\title{PhotoElectroChemistry in Corrosion}
\author{M. Skocic, PhD Electrochemistry and Materials}
\date{\vfill \includegraphics[width=0.70\textwidth]{full_bw.png}}

\begin{document}

\begin{frame}
    \titlepage
\end{frame}

\begin{frame}
    \frametitle{Contents}
    \tableofcontents
\end{frame}



\section{Introduction}
\begin{frame}{Introduction}
    \begin{itemize}
        \item Photoelectrochemical techniques have been shown to be useful tools for characterizing oxidation layers. 
        \item Interdisciplinary theoretical underpinnings were built \citep{morrison1980, vijh1969, stimming1986, diquarto1997, wouters2007} 
              such as the Gärtner-Butler model \citep{gaertner1959,butler1977}
              which has been proven to be a simple and robust model for the photocurrent generation. 
        \item Technical progresses were achieved, allowing to study oxide layers at 
              macroscopic, mesoscopic, and microscopic scales 
              \citep{benaboud2007, srisrual2011}, or in-situ in high temperature corrosion 
              conditions \citep{bojinov2002,skocic2016}.
    \end{itemize}
\end{frame}



\section{Basics}
\begin{frame}{Basics}
    Several hypotheses are needed in order to apply the theoretical concepts:  
    \begin{itemize}
        \item semiconductor are considered to be ideal i.e. crystallized and homogeneous  
        \item the dielectric constant of the semiconductor is independent of the light wavelength  
        \item the capacity of the Helmholtz layer is greater than the capacitance of the space charge capacitance  
        \item the potential drop in the Helmholtz layer is independent of the applied potential and is negligible
    \end{itemize}

    \tiny
    \begin{alertblock}{Warning}
        The hypotheses are rarely fully respected in the case of oxides or passive 
        films formed on industrial alloys. Nonetheless, the literature shows that the 
        developed models can be applied to non-ideal systems such as oxides 
        and passive films.
    \end{alertblock}
\end{frame}

\begin{frame}[allowframebreaks=1.00]{Electronic Band Structure}
    \input{./src/figures/tikz_band_model.tex}

    \input{./src/figures/tikz_excitation_carrier}

    \input{./src/figures/tikz_fermi_position}
\end{frame}


\section{Applications}

% BIBLIOGRAPHY
\begin{frame}[allowframebreaks=0.9]{References}
\AtNextBibliography{\tiny}
\nocite{*}
\printbibliography
\end{frame}

\end{document}
