% !TEX program = pdflatex
% !TEX encoding = UTF-8
% !TEX spellcheck = en_US

\documentclass[10pt, 3p, sort&compress]{elsarticle}
\journal{Journalof Nuclear Materials}
\bibliographystyle{elsarticle-harv}

\usepackage[english]{babel}
\usepackage{amsmath}
\usepackage{amsfonts} 
\usepackage{amssymb}
\usepackage{amsthm}
\usepackage{mathtools}
\usepackage{microtype}
\usepackage[utf8]{inputenc}
\usepackage{graphicx} 
\usepackage{caption}
\usepackage{subcaption}
\usepackage{float}

\graphicspath{{./figures/}}


\usepackage{booktabs}



\usepackage{multicol}
\usepackage{multirow}
\usepackage{url}
\usepackage{hyperref}

%edit the properties of your PDF documents which will be displayed
\hypersetup{
    bookmarks=true, 		% show bookmarks bar?
    unicode=true,  		% non-Latin characters in Acrobat’s bookmarks
    pdftoolbar=true,        % show Acrobat’s toolbar?
    pdfmenubar=true,        % show Acrobat’s menu?
    pdffitwindow=true,      % page fit to window when opened
    pdftitle={},    % title
    pdfauthor={},     % author
    pdfsubject={},   % subject of the document
    pdfnewwindow=true,      % links in new window
    pdfkeywords={}, % list of keywords
    colorlinks=false,       % false: boxed links; true: colored links
    linkcolor=red,          % color of internal links
    citecolor=green,        % color of links to bibliography
    filecolor=magenta,      % color of file links
    urlcolor=cyan           % color of external links
}


\newcommand{\modi}[1]{\vert #1 \vert}
\newcommand{\sigmae}{\sigma _{exp}}
\newcommand{\sigmap}{\sigma _{p}}
\newcommand{\iphe}{I_{ph,exp}}
\newcommand{\iphc}{I_{ph,calc}}
\newcommand{\gradX}{\nabla \chi ^2}
\newcommand{\je}{J _{\epsilon}}
\newcommand{\hv}{h  \nu}
\newcommand{\tvp}{t_{dof,P}}
\newcommand{\eps}{\epsilon}
\newcommand{\epsp}{\epsilon ^{\prime}}
\newcommand{\jep}{J _{\epsp}}
\newcommand{\coef}{0.1}


\begin{document}

\begin{frontmatter}

%\title{A Novel Approach to Analyze Photocurrent Energy Spectra: Estimation of the confidence intervals}
\title{PhotoElectroChemistry: Fitting Procedure}
	
\author[1]{Milan Skocic\corref{cor1}}%\fnref{fn1}}
\ead{milan.skocic@framatome.com}
\cortext[cor1]{Corresponding author}
%\fntext[fn1]{This is the first author footnote.}

\affiliation[1]{organization={Framatome Technical Center},
addressline={30 Bld de L'Industrie, Espace Magenta},
postcode={71205},
city={Le Creusot},
country={France}}
	
	
\begin{abstract}
The weighting terms for the numerical fitting procedure were defined using the definition of the $\chi ^2$ distribution and a scaling factor for the covariance matrix in order to correctly estimate the confidence intervals of fitted parameters. Moreover, the confidence intervals were helpful for determining the number of semiconductive contributions. Experimental spectra were successfully tested with up to 12 contributions over 3 different potentials.
\end{abstract}
	

\end{frontmatter}
	
\section{Introduction}

In the course of the last 30 years, photoelectrochemical techniques have been shown to be useful tools for characterizing oxidation layers. Interdisciplinary theoretical underpinnings were built \citep{morrison1980, vijh1969, stimming1986, diquarto1997, wouters2007} such as the Gärtne-Butler model \citep{gartner1959,butler1977} which has been proven to be a simple and robust model for the photocurrent generation. Technical progresses were achieved, allowing to study oxide layers at macroscopic, mesoscopic, and microscopic scales \citep{benaboud2007, srisrual2011}, or in-situ in high temperature corrosion conditions \citep{bojinov2002,skocic2016}.

Up to now, for complex oxide scales formed of several p-type and n-type phases, the complete description of the photocurrent energy spectra could not be achieved, and only semi-quantitative and/or partial information could be obtained on the nature of the phases present in the oxide layers. Recently, a new approach was proposed by \citet{petit2013} to analyze the photocurrent spectra which was applied to oxidized duplex stainless steels [9] as well as Ni-based and Zr-based alloys oxidized in LWR conditions \citep{skocic2016}. The numerical fitting procedure allowed to obtain high quality fits of the experimental data. Nonetheless, the estimation of the confidence intervals was not implemented.

This paper presents the additional work carried out in order to implement the estimation of the confidence intervals based on the fitting procedure developed by \citet{petit2013}. The latter was rewritten in Python which is an open source interpreted programming language. It is largely used in the scientific community \citep{langtangen2012, millman2011, kiusalaas2010, oliphant2007} and it comes with optimized libraries for numerical computations \citep{vanderwalt2011,jones2020} and a high quality 2D visualization library \citep{hunter2007}.


\section{Fitting procedure}
It should be reminded that the low photocurrents are extracted from the overall electrochemical current using a modulation of the light illuminating the sample. The modulation is obtained with a mechanical chopper placed on the optical path. The reference signal of the modulation is connected to the external reference input of a lock-in amplifier, whereas the current output is connected to the signal input of the latter, allowing to measure both the modulus, $\vert I_{ph} \vert$ , and the phase shift, $\theta$ of the so called as-measured photocurrent, $I_{ph}$. Then, the latter is converted, at each photon energy, in a more useful value, $I_{ph}^* (h\nu)$, proportional to the quantum yield of the photocurrent, by dividing $I_{ph}(h\nu)$ by $\Phi(h\nu)/\Phi_{max}$  , where $\Phi(h\nu)$ is the photon flux arriving onto the sample, and $\Phi_{max}$ its maximum value. 

As $I_{ph}^*$ is measured under modulated light conditions and thus actually was a complex number, it was proposed that the real part $\Re I_{ph}^* $ and the imaginary part $\Im I_{ph}^*$ of the photocurrent $I_{ph}^*$ should be considered simultaneously when analyzing and fitting the photocurrent energy spectra, rather than the modulus $\vert I_{ph}^* \vert$ only, as it was the case up to now. Therefore, the overall complex photocurrent, $I_{ph}^*$, was written as shown in eq. \ref{eq:Iph_complex}.

\begin{equation}
\begin{split}
I_{ph}^* &= \vert I_{ph}^* \vert \cdot \cos \theta + \imath \vert I_{ph}^* \vert \cdot \sin \theta \\
I_{ph}^* &= \sum _i \vert I_{ph,i}^* \vert \cdot \cos \theta _i + \sum _i \vert I_{ph,i}^* \vert \cdot \sin \theta _i 
\end{split}
\label{eq:Iph_complex}
\end{equation}

\noindent where $\vert I_{ph,i}^* \vert$ and $\theta _i$ represent the modulus and phase shift, respectively, of the photocurrent issued from the $i^{th}$ semiconducting constituent of the oxide layer. For thin semiconducting films such as those usually investigated in most corrosion studies, the space charge regions are low compared to penetration depth of the light.  $\vert I_{ph,i}^* \vert$ may thus be expected, at a given applied potential, to follow the simplified form of the Gärtner–Butler model, i.e. in fact to obey to the eq. \ref{eq:model}.

\begin{equation}
\begin{split}
\left( \vert I_{ph,i}^* \vert \cdot E \right)^{\frac{1}{n}} = K_i \cdot \left(  E - E_{g,i}  \right)
\end{split}
\label{eq:model}
\end{equation}

\noindent where $E_{g,i}$ and $K_i$ represent the energy gap and a proportionality value, respectively. It should be emphasized that the as-defined $\vert I_{ph,i}^* \vert$ is proportional to, but not equal, to the quantum yield for the ith semiconducting constituent. $n$ depends on the band to band transition type, $n = 1/2$ for an allowed direct transition, and $n = 2$ for an allowed indirect transition. To our knowledge, the case where $n = 1/2$ (direct transition) was rarely observed in the case of passive films or more or less disordered thin oxide films.

In addition, as the space charge regions are likely to extend over the whole thickness of each phase in the oxide layer, it is assumed that the recombination of the photogenerated electron---hole pairs, and thus the phase shifts, $\theta _i$, will not depend on the photon energy.
A given vector of $m$ $(E_{g,i}, K_i , \theta_i)$ triplets represents the supposed number of semiconducting phases contributing to the photocurrent. The scalar function, $D$, to be minimized is given in  eq. \ref{eq:D} which represents a measurement of the distance between the experimental and calculated data.

\begin{equation}
\begin{split}
\left( \vert I_{ph,i}^* \vert \cdot E \right)^{\frac{1}{n}} = K_i \cdot \left(  E - E_{g,i}  \right)
\end{split}
\label{eq:D}
\end{equation}

\section{Estimation of the confidence intervals}

The scalar function to be minimized defined in \ref{eq:D} used for fitting the photocurrent spectra ensures a fairly fast convergence towards the $3m$ parameters defining the semiconductive contributions. However, the as-defined scalar function could not be used for estimating the confidence intervals. An alternative scalar function was defined to be computed with the optimal parameters in order to estimate the confidence intervals. The statistics of curve fitting shows that the confidence intervals can be estimated using the least-squares method which can be applied to nonlinear systems \citep{bevington2003,nocedal2006}.

The least-squares regression uses the proprieties of the $\chi ^2$ distribution. Consequently, of the experimental measurements of the photocurrent spectra are assumed to follow the normal distribution. Moreover, the least-squares method can be strictly applied only when the experimental variances are known for each energy value of the photocurrent spectrum. Nonetheless, the latter are not always known as it is the case for the photoelectrochemical characterizations. Consequently, modifications of the relationships defined for the ideal situation were necessary.

\subsection{Ideal Situation}
The equation \ref{eq:chi2} presents the scalar function $\chi ^2$ defined in the least-squares method when the experimental variances $\sigmae ^2$ are known. The residuals, $\epsilon$, weighted by the inverse of variances are given by the equation \ref{eq:epsilon}. $\chi ^2$ is therefore defined as the sum of the weighted residuals as illustrated by the equation \ref{eq:chi2_epsilon}

\begin{equation}
\begin{split}
\chi ^2 = \sum _{\hv} \frac{\modi{\iphc-\iphe}^2}{\sigmae^2}
\end{split}
\label{eq:chi2}
\end{equation}

\begin{equation}
\begin{split}
\epsilon = \frac{\modi{\iphc-\iphe}^2}{\sigmae}
\end{split}
\label{eq:epsilon}
\end{equation}

\begin{equation}
\begin{split}
\chi ^2 = \sum _{\hv} \eps ^2
\end{split}
\label{eq:chi2_epsilon}
\end{equation}

$\gradX$ approaches zero when the parameter values are approaching the optimum values. The covariance matrix of the fitted parameters, $\sigmap ^2$, can be estimated with the Jacobian, $\je$, of the weighted residuals \citep{bevington2003} and its expression is given by the equation \ref{eq:sigma_p}. For nonlinear systems, as it is the case for the photocurrent, the equation \ref{eq:sigma_p} is a first-order approximation. In fact, the approximation is valid because the second-order terms are close to zero which avoids to compute the Hessian \citep{press2007}.

\begin{equation}
\begin{split}
\sigmap = ( \je ^T \cdot \je ) ^{-1}
\end{split}
\label{eq:sigma_p}
\end{equation}

The diagonal terms of the covariance matrix represent the variances of the parameters. The P\% confidence interval of the parameters, $CI_{P\%}$ is obtained by multiplying the standard deviations with the student coefficient, $\tvp$, with dof being the degree of freedom corresponding to number of experimental points of the photocurrent, N, minus the number of parameter, 3m. The probability $P\%$ was set to 95\%. The confidence intervals of the parameters are given by the equation \ref{eq:CIP}.

\begin{equation}
\begin{split}
CI_P &= \sqrt{ diag(\sigmap ^2)} \cdot \tvp \\
dof &= N-3m
\end{split}
\label{eq:CIP}
\end{equation}

\subsection{Real Situation}
The confidence interval can be estimated even when the experimental variances, $\sigmae ^2$, are not known. However, it is necessary to modify the equations presented in section 3.1. The objective is to define a scalar function, $S$, which behaves like $\chi ^2$ with a constant scaling factor $g$. The scalar function, $S$, is defined using real and positive weighting terms, $w$, as shown in equation \ref{eq:chi2_epsilon}. Similarly to the Equation 6, the $S$ function corresponds to the sum of the weighted residuals $\epsp$.

\begin{equation}
\begin{split}
S &= \sum _{\hv} \frac{\modi{\iphc-\iphe}^2}{w ^2} \\
%%\epsp &= \sum _{\hv} \frac{\modi{\iphc-\iphe}}{\sqrt{w}} \\
%S & = \sum _{\hv} \epsp ^2 
\end{split}
\label{eq:S}
\end{equation}

The weighting terms are defined in order to isolate the scaling factor, $g$, and the experimental variances such as illustrated by the equation \ref{eq:w}. Therefore, the weighting terms are considered proportional to the experimental variances.
\begin{equation}
	w=g \cdot \frac{1}{\sigmae ^2}
	\label{eq:w}
\end{equation}	
	
Combining equation \ref{eq:S} and equation \ref{eq:w}, the scalar function $S$ becomes proportional to $\chi ^2$. Moreover, $\frac{\chi ^2}{\nu}$ goes to unity when the optimum values of the parameters are reached. Consequently, the scaling factor $g$ can be computed with the optimum value of $S$ as shown by the equation \ref{eq:S_scaling_unity}.
\begin{equation}
\begin{split}
S &=g \cdot \chi ^2 \\
\frac{S}{\nu}&=g \cdot \frac{\chi ^2}{\nu} \simeq 1
\end{split}
\label{eq:S_scaling_unity}
\end{equation}
	

The covariance matrix, $\sigmap$, can therefore be estimated with the scaling factor $g$ and the Jacobian of the weighted residuals, $\jep$ as illustrated by the equation \ref{eq:sigma_p}. The confidence intervals are then computed using the Equation \ref{eq:CIP}.
	
\begin{equation}
\begin{split}
\sigmap ^{\prime 2} &= ( \jep ^T \cdot \jep ) ^{-1} \\
\sigmap &= g \cdot \sigmap ^2
\end{split}
\label{eq:sigma_p}
\end{equation}

The choice of the weighted terms was made considering that the experimental variances are proportional to average noise of the photocurrent modulus in dark conditions, $\bar{\epsilon _d}$. It has been proposed that the variances are smaller when the quantum yield is greater. The latter is represented by the photocurrent corrected by the photon flux normalized to its maximum value, $I_{ph,N}$. The normalization ensures that the weighting terms have the same dimension as the inverse of the variances as shown in equation \ref{eq:w_epsd}. $S$ is therefore adimensional as it is the case for $\chi ^3$. The Jacobian, $\jep$ is numerically estimated by fixing the finite difference step to the squared root of the machine precision \citep{nocedal2006, press2007}.

\begin{equation}
\begin{split}
\sigmae ^2 & \propto \frac{\bar{\epsilon _d}}{\vert I_{ph,N} \vert} \\
w = & \frac{1}{\sigmae ^2} = g \cdot \frac{\vert I_{ph,N} \vert ^2}{\bar{\epsilon _d} ^2}
\end{split}
\label{eq:w_epsd}
\end{equation}


\section{Application}
\subsection{Numerically generated energy photocurrent spectra}

The weighted terms, as previously defined, represent the signal/noise ratio. In order to test the relevancy of the weighted terms, energy photocurrent spectra were recomputed (eq. \ref{eq:Iph_complex}) from parameter values obtained by \citet{petit2013} by fitting a fairly simple energy photocurrent spectrum having 3 semiconductive contributions. The values of the parameters are presented in \ref{table:3m_params}.

\begin{table}[htpb]
\begin{tabular}{ p{1cm}|p{2cm}|p{2cm}| p{2cm}}
\toprule
 & $10^5 \ K_i$ & $\theta _i$ &  $E_{g,i}$\\
 & $A^{1/2} \cdot eV^{1/2}$ & ° & $eV$\\
\midrule
& 4.6    & 7.0 &   1.91\\
&   5.4  & -33.0 & 2.44\\
& 7.0 & 156.0 &  3.16\\
 \bottomrule
\end{tabular}
\caption{Parameter values obtained by \citet{petit2013} (figure 1) by numerical fitting.}
\label{table:3m_params}
\end{table}
 
 An increasing noise was added to the computed values of $I_{ph}$. The noise was calculated using a normal centered distribution $N(0,\sigma)$ where $\sigma$ was set to the minimal value of the calculated photocurrent et then amplified using an amplification factor $f_a$. The generated noise was added to real and imaginary parts of the photocurrent $I_{ph}$. The modules of the corrected photocurrent $I_{ph}^*$ for different amplification factors are shown in figure \ref{fig:data_noise} and the corresponding values of the fitted parameters are presented in table \ref{table:result_fit_noise}. The confidence intervals increased when the signal/noise ratio decreased as it was expected.
 
\renewcommand{\coef}{0.3}
\begin{figure*}[htbp]
	\centering
	\begin{subfigure}{\coef\textwidth}
		\centering
	 	\includegraphics[width=\textwidth]{DSS_0mV_data-Iph-0001x.png}
	 	\caption{$f_a=0.001$}
	 	\label{fig:fa0001}
	\end{subfigure}
	\begin{subfigure}{\coef\textwidth}
		\centering
	 	\includegraphics[width=\textwidth]{DSS_0mV_data-Iph-1x.png}
	 	\caption{$f_a=1$}
	 	\label{fig:fa1}
	\end{subfigure}
	\begin{subfigure}{\coef\textwidth}
		\centering
	 	\includegraphics[width=\textwidth]{DSS_0mV_data-Iph-2x.png}
	 	\caption{$f_a=2$}
	 	\label{fig:fa2}
	\end{subfigure}
	
	\begin{subfigure}{\coef\textwidth}
		\centering
	 	\includegraphics[width=\textwidth]{DSS_0mV_data-Iph-5x.png}
	 	\caption{$f_a=5$}
	 	\label{fig:fa5}
	\end{subfigure}\quad
	\begin{subfigure}{\coef\textwidth}
		\centering
	 	\includegraphics[width=\textwidth]{DSS_0mV_data-Iph-10x.png}
	 	\caption{$f_a=10$}
	 	\label{fig:fa10}
	\end{subfigure}
	\caption{Energy photocurrent spectra generated with different amplification factors $f_a$.}
	\label{fig:data_noise}
\end{figure*}

\renewcommand{\coef}{0.32}
\begin{figure*}[htpb]
	\centering
	\begin{subfigure}{\coef\textwidth}
		\centering
	 	\includegraphics[width=\textwidth]{Anusara-690-0mV.png}
	 	\caption{}
	 	\label{fig:data_srisrual1}
	\end{subfigure}
	\begin{subfigure}{\coef\textwidth}
		\centering
	 	\includegraphics[width=\textwidth]{Anusara-690-300mV.png}
	 	\caption{}
	 	\label{fig:data_srisrual2}
	\end{subfigure}
	\begin{subfigure}{\coef\textwidth}
		\centering
	 	\includegraphics[width=\textwidth]{Anusara-690-600mV.png}
	 	\caption{}
	 	\label{fig:data_srisrual3}
	\end{subfigure}
	
	\caption{Energy photocurrent spectra recorded at different applied potentials on a Ni-based alloy A600 oxidized at 900°C in oxygen for 2h (according to \citep{srisrual2013}).}
	\label{fig:data_srisrual}
\end{figure*}

\begin{table*}[htpb]
\tiny
\begin{tabular}{ p{0.5cm}|p{1.2cm}p{1.2cm}p{1.3cm}|p{1cm}p{0.8cm}p{1.1cm}|p{1cm}p{0.8cm}p{1.1cm}|p{0.9cm}p{0.8cm}p{1.1cm}}
\toprule
 $U$   & $10^5 \ K_i$ & $\theta _i$ &  $E_{g,i}$ & $10^5 \ K_i$ & $\theta _i$ &  $E_{g,i}$ & $10^5 \ K_i$ & $\theta _i$ &  $E_{g,i}$ & $10^5 \ K_i$ & $\theta _i$ &  $E_{g,i}$\\
 $mV$ & $A^{1/2} \cdot eV^{1/2}$ & ° & $eV$ & $A^{1/2} \cdot eV^{1/2}$ & ° & $eV$ & $A^{1/2} \cdot eV^{1/2}$ & ° & $eV$ & $A^{1/2} \cdot eV^{1/2}$ & ° & $eV$\\
\midrule
100   & $5.19 \pm 0.09$ & $-42.6 \pm 0.5$  & $1.74 \pm 0.02$ 
         & $6.4 \pm 0.4$ & $122 \pm 2$ & $2.42 \pm 0.04$ 
         & $6.5 \pm 0.4$ & $134 \pm 4$ & $2.88 \pm 0.05$ 
         & $8.9 \pm 0.8$ & $-64 \pm 6$ & $3.47 \pm 0.06$\\

\midrule
0   & $5.14 \pm 0.07$ & $-49.2 \pm 0.5$  & $1.755 \pm 0.008$ 
     & $6.1 \pm 0.3$ & $120 \pm 2$ & $2.41 \pm 0.04$ 
     & $6.8 \pm 0.4$ & $131 \pm 4$ & $2.82 \pm 0.04$ 
     & $9.1 \pm 0.9$ & $-58 \pm 6$ & $3.48 \pm 0.06$\\

\midrule
-100   & $4.7 \pm 0.1$ & $-52.4 \pm 0.5$  & $1.76 \pm 0.02$ 
         & $6.1 \pm 0.3$ & $119 \pm 2$ & $2.44 \pm 0.03$ 
         & $6.9 \pm 0.4$ & $131 \pm 4$ & $2.91 \pm 0.04$ 
         & $9.2 \pm 0.9$ & $-56 \pm 6$ & $3.43 \pm 0.06$\\
         
\midrule
-200   & $4.01 \pm 0.06$ & $-53.5 \pm 0.6$  & $1.76 \pm 0.02$ 
         & $5.1 \pm 0.4$ & $121 \pm 3$ & $2.43 \pm 0.04$ 
         & $6.1 \pm 0.4$ & $124 \pm 4$ & $2.85 \pm 0.04$ 
         & $8.3 \pm 0.7$ & $-63 \pm 6$ & $3.46 \pm 0.06$\\
         
\midrule
-300   & $2.9 \pm 0.3$ & $-52 \pm 2$  & $1.76 \pm 0.05$ 
         & $4.2 \pm 0.6$ & $122 \pm 5$ & $2.42 \pm 0.09$ 
         & $5.7 \pm 0.5$ & $122 \pm 4$ & $2.82 \pm 0.08$ 
         & $7.6 \pm 0.3$ & $-64 \pm 3$ & $3.43 \pm 0.06$\\
         
\midrule
-400   & $1 \pm 2$ & $-50 \pm 30$  & $1.7 \pm 0.6$ 
         & $4 \pm 3$ & $120 \pm 20$ & $2.4 \pm 0.5$ 
         & $5 \pm 2$ & $130 \pm 20$ & $2.8 \pm 0.3$ 
         & $6.7 \pm 0.7$ & $-61 \pm 6$ & $3.35 \pm 0.08$\\


\bottomrule
\end{tabular}
\caption{Parameters values and the associated confidence intervals obtained after numerical fitting of the energy photocurrent spectra of the figure \ref{fig:exp_iph_fit}. The potential is referred with respect to mercury sulfate electrode (MSE, +650 V vs. SHE).}
\label{table:exp_iph_fit}
\end{table*}



\begin{table}[thbp]
\tiny
\begin{tabular}{ p{1cm}|p{2.cm}|p{2.cm}| p{2.cm}}
\toprule
 $f_a$ & $10^5 \ K_i$ & $\theta _i$ &  $E_{g,i}$\\
 & $A^{1/2} \cdot eV^{1/2}$ & ° & $eV$\\
\midrule
\multirow{3}{*}{0.0001} & $4.6000 \pm 0.0002$    & $7.00 \pm 0.02$ &   $1.9100 \pm 0.0002$\\
                                   & $5.4000 \pm 0.0004$    & $-33.00 \pm 0.02$ &   $2.4400 \pm 0.0003$\\
                                   & $7.0000 \pm 0.0009$    & $156.00 \pm 0.02$ &   $3.1600 \pm 0.0009$\\
                                   
\midrule
\multirow{3}{*}{1}        & $4.6 \pm 0.2$    & $-7 \pm 4$ &   $1.91 \pm 0.04$\\
                                   & $5.5 \pm 0.4$    & $-33 \pm 5$ &   $2.45 \pm 0.08$\\
                                   & $7.0 \pm 0.3$    & $156 \pm 4$ &   $3.15 \pm 0.04$\\
                                  
\midrule
\multirow{3}{*}{2}        & $4.6 \pm 0.6$    & $-7 \pm 7$ &   $1.91 \pm 0.09$\\
                                   & $5.5 \pm 0.7$    & $-32 \pm 20$ &   $2.4 \pm 0.2$\\
                                   & $7.0 \pm 0.5$    & $160 \pm 6$ &   $3.2 \pm 0.2$\\      
                                   
\midrule
\multirow{3}{*}{5}        & $5 \pm 3$    & $8 \pm 30$ &   $1.9 \pm 0.4$\\
                                   & $5 \pm 3$    & $-38 \pm 50$ &   $2.4 \pm 0.5$\\
                                   & $7 \pm 3$    & $154 \pm 20$ &   $3.2 \pm 0.2$\\        
                                   
\midrule
\multirow{3}{*}{10}        & $5 \pm 8$    & $10 \pm 80$ &   $1.9 \pm 0.9$\\
                                   & $5 \pm 6$    & $-43 \pm 200$ &   $2 \pm 2$\\
                                   & $6 \pm 4$    & $-155 \pm 60$ &   $3.1 \pm 0.7$\\                                   


\bottomrule
\end{tabular}
\caption{Parameter values obtained by \citet{petit2013} (figure 1) by numerical fitting.}
\label{table:result_fit_noise}
\end{table}

 
The estimation of the confidence intervals can also be helpful for determining the number of semiconductive contribution in an energy photocurrent spectrum. In fact, the determination of the number of semiconductive contributions is an iterative operation by adding contributions until the spectrum is correctly fitted. The estimation of the confidence intervals can be used as a break point of the iterative search when the intervals of two contributions are overlapping i.e. they are no more statistically discernable. 

For the sake of illustration, the energy photocurrent of the figure \ref{fig:fa0001} was fitted by considering 3, 4 and 5 semiconductive contributions for energies ranging from 1.8 eV to 4.0 eV. Table \ref{table:result_noise_contributions} shows the bandgap values and the associated confidence intervals obtained after numerical fitting. 

\begin{table}[htpb]
\small
\begin{tabular}{ p{1cm}|p{2cm}|p{2cm}| p{2cm}}
\toprule
 $m$ & 3 & 4 &  5\\
\midrule
$E_{g,1}$ & $1.91 \pm 0.07$    & $1.9 \pm 0.4$      & $1.9 \pm 0.1$ \\
$E_{g,2}$ &  $2.4 \pm 0.2$      & $2 \pm 5$           & $2 \pm 4$\\
$E_{g,3}$ & $3.16 \pm 0.06$    & $2.5 \pm 0.4$     &  $2.5 \pm 0.2$\\
$E_{g,4}$ &                           &  $3.16 \pm 0.06$ &   $2.7 \pm 0.2$    \\
$E_{g,5}$ &                           &                         &    $3.2 \pm 0.1$     \\
 \bottomrule
\end{tabular}
\caption{Bandgap values and the associated confidence intervals obtained after numerical fitting of the energy photocurrent spectra of the figure \ref{fig:fa0001}.}
\label{table:result_noise_contributions}
\end{table}

The numerical fitting, by considering 4 contributions, showed that the contribution with a bandgap value of 1.91 
eV was split into two contributions (1.9 eV and 2 eV). The second one featured a confidence interval in the same 
order of magnitude as the value itself i.e. the 4th contribution did not improve the fitting of the experimental 
data. 
The numerical fitting, by considering 5 contributions, showed that the same splitting of the contribution with a 
bandgap value of 1.91 eV. Moreover, the contribution with a bandgap value of 2.4 eV was split into two 
contributions (2.6 eV and 2.7 eV) whose confidence intervals were overlapping indicating that they were not 
statically discernable.

\subsection{Experimental energy photocurrent spectra}
The defined procedure for estimating the interval confidences was then applied to energy photocurrent spectra recorded at different potentials on a Ni-based alloy 600 thermally oxidized. The experimental data were provided by \citet{petit2013}. The experimental, as well as the fitted, modules of the corrected photocurrent $\vert I_{ph}^* \vert$ are illustrated in figure \ref{fig:exp_iph_fit}. 

The photocurrent modulus in dark conditions, $\bar{\epsilon _d}$, was computed by taking the average of the photocurrent modulus for the five highest energies (6.19, 6.17, 6.14 and 6.08 eV) where the emission spectrum of a Xe lamp can be reasonably considered close to zero. The photocurrent modulus featured a strong decrease for energies lower than 3 eV when the potential decreased towards more cathodic values indicating that the ratio signal/noise decreased as well. 

Table \ref{table:exp_iph_fit} shows the parameters and the associated confidence intervals obtained after numerical fitting of the experimental data. The increase of the computed confidence intervals for the three first contributions, having bandgap values lower than 3 eV, mirrored correctly the decrease of the ratio signal/noise observed on the experimental data in figure \ref{fig:exp_iph_fit}.

\begin{figure}[htbp]
\centering
\includegraphics[width=0.4\textwidth]{Abdel-600-All-Ipht.png}
\caption{Energy photocurrent spectra recorded at different potentials on a Ni-based alloy 600 thermally oxidized (experimental data were provided by \citet{petit2013})}
\label{fig:exp_iph_fit}
\end{figure}



The fitting procedure was also applied to additional photocurrent spectra obtained by \citet{srisrual2013} where up to 12 contributions were found to be statistically discernable over 3 different potentials as illustrated in figure \ref{fig:data_srisrual} and table \ref{table:data_srisrual}.
\begin{table}[h]
\tiny
\begin{tabular}{ p{1cm}|p{2cm}p{2cm} p{2cm}}
\toprule
                & $0 mV_{MSE}$ & $-300 mV_{MSE}$ & $-600 mV_{MSE}$ \\
\midrule
$E_{g,1}$   & $1.7 \pm 0.2$ & $2 \pm 3$ & $2 \pm 20$ \\
$E_{g,2}$   & $2.0 \pm 0.2$ & $2 \pm 3$ & $2 \pm 8$ \\
$E_{g,3}$   & $2.25 \pm 0.09$ & $2 \pm 1$ & $2 \pm 4$ \\
$E_{g,4}$   & $2.58 \pm 0.04$ & $2.6 \pm 0.3$ & $3 \pm 2$ \\
$E_{g,5}$   & $2.8 \pm 0.04$ & $2.9 \pm 0.2$ & $2.8 \pm 0.9$ \\
$E_{g,6}$   & $2.96 \pm 0.02$ & $3.08 \pm 0.01$ & $3.09 \pm 0.05$ \\
$E_{g,7}$   & $3.077 \pm 0.0022$ & $3.16 \pm 0.03$ & $3.2 \pm 0.2$ \\
$E_{g,8}$   & $3.195 \pm 0.003$ & $3.19 \pm 0.02$ & $3.2 \pm 0.05$ \\
$E_{g,9}$   & $3.27 \pm 0.02$ & $3.42 \pm 0.03$ & $3.42 \pm 0.04$ \\
$E_{g,10}$   & $3.44 \pm 0.03$ & $4.073 \pm 0.009$ & $4.047 \pm 0.008$ \\
$E_{g,11}$   & $3.8 \pm 0.3$ & $4.7 \pm 0.1$ & $4.7 \pm 0.2$ \\
$E_{g,12}$   & $4.1 \pm 0.5$ &  &  \\
     
\bottomrule
\end{tabular}
\caption{Parameters values and the associated confidence intervals obtained after numerical fitting of the energy photocurrent spectra of the figure \ref{fig:data_srisrual}}
\label{table:data_srisrual}
\end{table}

\subsection{Conclusion} 

The weighting terms for the numerical fitting procedure were defined using the definition of the $\chi ^2$ distribution and a scaling factor for the covariance matrix.
The latter correctly reflected the noise of the experimental data in the computed i.e. the covariance matrix will correctly estimate the confidence intervals for the fitted parameters. 

Moreover, the confidence intervals were helpful for determining the number of semiconductive contributions. The estimation of the confidence intervals can be used as a break point of the iterative search when the intervals of two contributions are overlapping i.e. they are no more statistically discernable. 

Experimental spectra were tested with up to 12 contributions over 3 different potentials and the estimated confidence intervals were helpfull at asserting that the semiconductive contributions are statistically discernable.


\bibliography{references}


\end{document}