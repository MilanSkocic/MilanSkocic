\section{Introduction}
In the course of the last 30 years, photoelectrochemical techniques have been 
shown to be useful tools for characterizing oxidation layers. 
Interdisciplinary theoretical underpinnings were built 
\citep{morrison1980-1, vijh1969-1, stimming1986-1, diquarto1997-1, wouters2007-1} 
such as the Gärtne-Butler model \citep{gartner1959-1,butler1977-1} which has been 
proven to be a simple and robust model for the photocurrent generation. 
Technical progresses were achieved, allowing to study oxide layers at 
macroscopic, mesoscopic, and microscopic scales 
\citep{benaboud2007-1, srisrual2011-1}, or in-situ in high temperature corrosion 
conditions \citep{bojinov2002-1,skocic2016-1}.

First, this paper presents the theoretical background on which the 
photoelectrochemical techniques rely on. Examples of application are 
also presented in a second part.