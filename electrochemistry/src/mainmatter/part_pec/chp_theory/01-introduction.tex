\section{Introduction}
    In the course of the last 30 years, photoelectrochemical techniques have been 
    shown to be useful tools for characterizing oxidation layers. 
    Interdisciplinary theoretical underpinnings were built 
    \citep{morrison1980, vijh1969, stimming1986, diquarto1997, wouters2007} 
    such as the Gärtne-Butler model\index{Gärtner-Butler} 
    \citep{gartner1959,butler1977} which has been proven to be a simple and 
    robust model for the photocurrent generation. 
    Technical progresses were achieved, allowing to study oxide layers at 
    macroscopic, mesoscopic, and microscopic scales 
    \citep{benaboud2007, srisrual2011}, or in-situ in high temperature corrosion 
    conditions \citep{bojinov2002,skocic2016}.

    First, this paper presents the theoretical background on which the 
    photoelectrochemical techniques rely on. Examples of application are 
    also presented in a second part.
