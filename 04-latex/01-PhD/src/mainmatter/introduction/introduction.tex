% !TEX root = ../../main_file.tex
% !TeX TS-program = pdflatex
% !TeX checkspelling = fr-toutesvariantes

\chapter*{Introduction}\label{chap:introduction}
\phantomsection
\addstarredchapter{Introduction}
%\begin{refsection}

Dans les réacteurs nucléaires, la première barrière de confinement du combustible est assurée par la gaine en alliage 
de zirconium. La gaine représente donc un enjeu capital en termes de durabilité face à l'environnement agressif
d'un coeur de réacteur. Le vieillissement, et plus précisément l'oxydation des gaines, est un facteur de limitation de la durée de
vie des assemblages de combustible.

Ce travail de thèse, réalisée dans le cadre d'une collaboration entre le laboratoire
SIMaP (UMR CNRS 5266/Grenoble INP/UGA) et le groupe AREVA, porte sur l'étude expérimentale d'un phénomène de corrosion local sur la gaine
communément appelé \emph{Shadow Corrosion}. Le phénomène de \emph{Shadow Corrosion} est observé dans les réacteurs à eau
bouillante et se traduit par une augmentation de l’épaisseur d’oxyde sur des zones de la gaine qui sont à
proximité d’autres éléments de l’assemblage constitués de matériaux différents tels que des aciers inoxydables et des
alliages à base de nickel (Inconel). 

Ce phénomène n'a jamais pu être reproduit en laboratoire.
Cependant, des études en réacteur test, réalisées dans les années 2000, ont permis de mettre en évidence les principaux
facteurs susceptibles d'influencer le mécanisme du phénomène de \emph{Shadow Corrosion}. Plus récemment, un travail de thèse
portant sur la modélisation  de ce phénomène a été réalisé dans le cadre d'une collaboration entre le laboratoire SIMaP
et le groupe AREVA. L'objectif principal en était de développer des modèles numériques de
couplage galvanique entre alliages de zirconium et différents matériaux, et de l'oxydation du zirconium, ces modèles nécessitant bien sûr des données d'entrée expérimentales pour être validés et calibrés.

Le présent travail de thèse fait suite à ce travail de modélisation et avait pour objectif principal de contribuer à comprendre
 le mécanisme de \emph{Shadow Corrosion} en tentant de le reproduire au laboratoire et en l'étudiant par des techniques 
(photo-)électrochimiques.
Pour ce faire, il était nécessaire de développer une nouvelle cellule électrochimique couplable avec une boucle 
de contrôle de la chimie d'un réacteur à eau bouillante, et pourvue de fenêtres optiques.
L'étude expérimentale présentée dans ce mémoire se décline
selon deux axes majeurs: \emph{conception, développement et validation du dispositif expérimental} et \emph{approche expérimentale par
(photo-)électrochimie du phénomène de Shadow Corrosion}.

Le premier chapitre est une analyse bibliographique des points clés de l'oxydation du zirconium, complétée par celle de
résultats d'études expérimentales menées en réacteur de recherche, en particulier autour de la \emph{Shadow Corrosion}. 
Ces analyses ont permis d'orienter notre étude sur des points jugés comme des points clés.

Les principales caractéristiques des matériaux étudiés et les techniques 
expérimentales mises en oeuvre sont décrites dans un second chapitre, en particulier les méthodes électrochimiques classiques
 ainsi que les protocoles d'oxydation.

Le troisième chapitre aborde le premier axe majeur de notre travail en présentant les aspects particuliers de la photoélectrochimie. 
Ce chapitre est complété par
la description des différentes étapes de conception et de développement du dispositif expérimental, 
par la présentation des premiers résultats obtenus, et de l'amélioration que nous proposons en termes d'analyses des 
spectres en énergie de photocourant.

Les résultats expérimentaux obtenus avec la cellule développée pour ce travail sont rassemblés et discutés 
dans un quatrième et dernier chapitre, en lien avec le phénomène de \emph{Shadow Corrosion}.


%\printbibliography[heading=subbibintoc]
%\end{refsection}


