\section{Fitting procedure}
It should be reminded that the low photocurrents are extracted from the overall 
electrochemical current using a modulation of the light illuminating the sample. 
The modulation is obtained with a mechanical chopper placed on the optical path. 
The reference signal of the modulation is connected to the external reference 
input of a lock-in amplifier, whereas the current output is connected to the 
signal input of the latter, allowing to measure both the modulus, 
$\vert I_{ph} \vert$ , and the phase shift, $\theta$ of the so called 
as-measured photocurrent, $I_{ph}$. Then, the latter is converted, at 
each photon energy, in a more useful value, $I_{ph}^* (h\nu)$, proportional 
to the quantum yield of the photocurrent, by dividing $I_{ph}(h\nu)$ 
by $\Phi(h\nu)/\Phi_{max}$  , where $\Phi(h\nu)$ is the photon flux arriving 
onto the sample, and $\Phi_{max}$ its maximum value. 

As $I_{ph}^*$ is measured under modulated light conditions and thus actually 
was a complex number, it was proposed that the real part $\Re I_{ph}^* $ 
and the imaginary part $\Im I_{ph}^*$ of the photocurrent $I_{ph}^*$ should 
be considered simultaneously when analyzing and fitting the photocurrent 
energy spectra, rather than the modulus $\vert I_{ph}^* \vert$ only, as it 
was the case up to now. 
Therefore, the overall complex photocurrent, $I_{ph}^*$, was written as 
shown in eq. \ref{eq:Iph_complex}.

\begin{equation}
\begin{split}
I_{ph}^* &= \vert I_{ph}^* \vert \cdot \cos \theta + \imath \vert I_{ph}^* \vert \cdot \sin \theta \\
I_{ph}^* &= \sum _i \vert I_{ph,i}^* \vert \cdot \cos \theta _i + \sum _i \vert I_{ph,i}^* \vert \cdot \sin \theta _i 
\end{split}
\label{eq:Iph_complex}
\end{equation}

\noindent where $\vert I_{ph,i}^* \vert$ and $\theta _i$ represent the modulus 
and phase shift, respectively, of the photocurrent issued from the $i^{th}$ 
semiconducting constituent of the oxide layer. For thin semiconducting films 
such as those usually investigated in most corrosion studies, the space charge 
regions are low compared to penetration depth of the light.  
$\vert I_{ph,i}^* \vert$ may thus be expected, at a given applied potential, 
to follow the simplified form of the Gärtner–Butler model, i.e. in fact to 
obey to the eq. \ref{eq:model}.

\begin{equation}
\begin{split}
\left( \vert I_{ph,i}^* \vert \cdot E \right)^{\frac{1}{n}} = K_i \cdot \left(  E - E_{g,i}  \right)
\end{split}
\label{eq:model}
\end{equation}

\noindent where $E_{g,i}$ and $K_i$ represent the energy gap and a proportionality 
value, respectively. It should be emphasized that the as-defined 
$\vert I_{ph,i}^* \vert$ is proportional to, but not equal, to the quantum 
yield for the ith semiconducting constituent. $n$ depends on the band to band 
transition type, $n = 1/2$ for an allowed direct transition, and $n = 2$ for 
an allowed indirect transition. To our knowledge, the case where $n = 1/2$ 
(direct transition) was rarely observed in the case of passive films or more 
or less disordered thin oxide films.

In addition, as the space charge regions are likely to extend over the whole 
thickness of each phase in the oxide layer, it is assumed that the recombination 
of the photogenerated electron---hole pairs, and thus the phase shifts, 
$\theta _i$, will not depend on the photon energy.
A given vector of $m$ $(E_{g,i}, K_i , \theta_i)$ triplets represents the s
upposed number of semiconducting phases contributing to the photocurrent. 
The scalar function, $D$, to be minimized is given in  eq. \ref{eq:D} which 
represents a measurement of the distance between the experimental and calculated 
data.

\begin{equation}
\begin{split}
\left( \vert I_{ph,i}^* \vert \cdot E \right)^{\frac{1}{n}} = K_i \cdot \left(  E - E_{g,i}  \right)
\end{split}
\label{eq:D}
\end{equation}
